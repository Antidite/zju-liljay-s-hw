\documentclass[UTF8]{ctexart}
\usepackage{geometry}
\geometry{margin=1.5cm, vmargin={0pt,1cm}}
\setlength{\topmargin}{-1cm}
\setlength{\paperheight}{29.7cm}
\setlength{\textheight}{25.3cm}
\setlength{\parindent}{2em}
% useful packages.
\usepackage{amsfonts}
\usepackage{amsmath}
\usepackage{amssymb}
\usepackage{amsthm}
\usepackage{enumerate}
\usepackage{graphicx}
\usepackage{multicol}
\usepackage{fancyhdr}
\usepackage{layout}
\usepackage{listings}
\usepackage{float, caption}

\lstset{
    basicstyle=\ttfamily, basewidth=0.5em
}

% some common command
\newcommand{\dif}{\mathrm{d}}
\newcommand{\avg}[1]{\left\langle#1 \right\rangle}
\newcommand{\difFrac}[2]{\frac{\dif#1}{\dif#2}}
\newcommand{\pdfFrac}[2]{\frac{\partial#1}{\partial#2}}
\newcommand{\OFL}{\mathrm{OFL}}
\newcommand{\UFL}{\mathrm{UFL}}
\newcommand{\fl}{\mathrm{fl}}
\newcommand{\op}{\odot}
\newcommand{\Eabs}{E_{\mathrm{abs}}}
\newcommand{\Erel}{E_{\mathrm{rel}}}

\begin{document}

\pagestyle{fancy}
\fancyhead{}
\setlength{\headheight}{12.64723pt}
\lhead{李伊健, 3230102477}
\chead{数据结构与算法第七次作业}
\rhead{December.1st, 2024}

\section{Heapsort的实现}

\paragraph{
首先想要实现HeapSort大致需要实现一下几个功能: 
makeheap,
heapnode,
deleteMin}
\subsection{heapnode}
\paragraph{\hspace{2em}为了实现heapnode,我才用了在一个node上,使用比较的方法,找出最小的一个元素
并用其与node的root,swap保证最小堆的实现,再进行递归处理保证了交换后的部分以下仍是最小堆,
至多进行$O(\log n)$.}
\subsection{makeheap}
\paragraph{\hspace{2em}makeheap在对input.size进行检验是否小于等于1后,只需要着重讨论大于1的部分
此时我们只需要从input.size/2-1处开始往下遍历,因为后一半的元素是没有children的.
递归结束便完成了makeheap.此时遍历后时间复杂度是$O(n\log n)$.}
\subsection{deleteMin}
\paragraph{\hspace{2em}deleteMin实际上就是将root元素剥离然后重构heap,但为了保证in-place,我们
需要将Input delete的内存给再度利用,所以我选择使用swap第一个和最后一个元素,最后再对
new root进行堆化,也就是heapnode,遍历整个heap后,也完成重建heap.}

\section{Check的实现}
\paragraph{\hspace{2em}实际上也就是利用std::chrono::highresolutionclock函数进行运行时间的测量,
同时为了保证公平性,在开始check前,我对input进行了复制,让它们进行相同的数列排序,
顺序,逆序,随机,和重复四种测试,前两种我选择使用一次函数生成,后两种我选择用STL random
库中的函数进行生成,并对元素范围进行限制,前者范围远大于input.size以保证随机,后者远小于size
也大于1000,保证可重复性.最后再利用issort函数,对myheapsort进行检查,错误则输出wrong.
}
\section{结果及分析}
\paragraph{\hspace{2em}我运行了数次之后,取平均值之后并保留三位小数得到:\newline
}

\par

\begin{tabular}{|c|c|c|} 
    \hline  % 表格上方的横线
     & my heapsort time & std::sort\_heap time \\  % 第一行
    \hline  % 行与行之间的横线
    random sequence & 0.100s & 0.060s \\  % 第二行
    \hline
    ordered sequence & 0.046s & 0.028s \\  % 第三行
    \hline  % 表格底部的横线
    reverse sequence & 0.045s & 0.057s \\
    \hline
    repetitive sequence & 0.082s & 0.056s \\
    \hline
\end{tabular}


\par

\paragraph{\hspace{2em}不难发现,除了reverse sequence,my sortheap都比std效率低
(因为myheapsort是最小堆,最后实现的是逆序,于是我已经将mysortheap结果对调,以保证worst与best
情况对应)我猜测,std存在一些优化策略,比如随机化:
探测若干项是否满足序关系,如果的确如此,那么正如ordered所显示的那样,时间会大大缩短
。相反如果不满足序关系占大多数,std可能会先对此进行一个随机化操作,此时时间复杂度
是$O(1)$不会对整体复杂度产生太大影响。因此std的random,repetitive,reverse的时间
近乎于相等,而我的程序却没有对此的优化,导致不同程序之间运算时间相差过大.除了best情况
剩下的均比std要慢。通过查询资料我了解到,std在运算时可能有并行化向量化减少运算时间,而且
同时优化内存访问模式来提高缓存命中率,这些可能也是std效率比myheapsort快的原因。}

\end{document}
