\documentclass[UTF8]{ctexart}
\usepackage{geometry}
\geometry{margin=1.5cm, vmargin={0pt,1cm}}
\setlength{\topmargin}{-1cm}
\setlength{\paperheight}{29.7cm}
\setlength{\textheight}{25.3cm}
\setlength{\parindent}{2em}
% useful packages.
\usepackage{amsfonts}
\usepackage{amsmath}
\usepackage{amssymb}
\usepackage{xcolor}
\usepackage{amsthm}
\usepackage{enumerate}
\usepackage{graphicx}
\usepackage{multicol}
\usepackage{fancyhdr}
\usepackage{layout}
\usepackage{listings}
\usepackage{float, caption}

\lstset{
    basicstyle=\ttfamily, basewidth=0.5em
}

% some common command
\newcommand{\dif}{\mathrm{d}}
\newcommand{\avg}[1]{\left\langle#1 \right\rangle}
\newcommand{\difFrac}[2]{\frac{\dif#1}{\dif#2}}
\newcommand{\pdfFrac}[2]{\frac{\partial#1}{\partial#2}}
\newcommand{\OFL}{\mathrm{OFL}}
\newcommand{\UFL}{\mathrm{UFL}}
\newcommand{\fl}{\mathrm{fl}}
\newcommand{\op}{\odot}
\newcommand{\Eabs}{E_{\mathrm{abs}}}
\newcommand{\Erel}{E_{\mathrm{rel}}}

\begin{document}

\pagestyle{fancy}
\fancyhead{}
\setlength{\headheight}{12.64723pt}
\lhead{李伊健, 3230102477}
\chead{数据结构与算法项目作业——四则运算器}
\rhead{December.20th, 2024}

\section{expression\_evaluator的功能及实现}

\paragraph{
    \textcolor{red}{请注意测试时请不要出现空格}\\
首先我的expression\_evaluator实现了以下几个所要求的基本功能: \\
- 支持多重括号和四则运算。\\
- 支持有限位小数运算,但可以不考虑负数作为输入。\\
- 识别非法的表达式,如括号不匹配、运算符连续使用、表达式以运算符开头或结尾以及除数是 0 等。\\
- 能处理含有加(+)、减(-)、乘(*)、除(/)和括号(())的中缀表达式。\\
再者额外的一些功能:\\
- 考虑了负数,比如:1+-2.1是合法的,但 1++2.1 是非法的。\\
- 考虑了科学计数法,比如:-1+2e2是合法的。(注:E也是可以的下同)\\
- 考虑了科学计数法正负号,比如:-1+2e+2,-1+2e-2是合法的。(注:2e是非法的)\\
- 考虑了\{ \},[ ]匹配问题,但像\{ ( \} )是非法的。\\
- 考虑了各种不同的非法情况,并告诉用户何种错误情况。\\
}
\subsection{前言}
\paragraph{\hspace{2em}
虽然王老师所讲为将中缀表达式转化为后缀表达式,同时利用stack去储存运算,但我觉得在实现
上可能并不如List来的直观,这也是我如何去实现以及为什么使用List的原因。
\textcolor{red}{不过遗憾的是,在书写报告时我才发现这种算法在实现类似(2+3)*(1+2)类型时,即多非包含
括号时会出现一些问题,时间匆忙,因此最后我还是以栈来完成了这一部分的缺漏,为了方便代码我也选择保留,
也是希望以后有机会能对其再度完善。}
}
\subsection{思路简介}
\paragraph{\hspace{2em}
首先我初步的想法是去递归运算,如何递归呢?我想到在没有括号的情况下,问题将会被
大大简化,从而能被轻松解决。因此我设想一个List,能够在遇到从左遇到第一个左括号时,
再去寻找一个与之匹配的右括号,将括号外部分复制入List,同时预留一个subval node,
括号内算式的存入sublist,递归下去,最终必将不含括号,然后将childlist运算的值后返回parentList的subval node,
从而List中不存在括号,可以运算,逐次返回childList的val,最终至List。return List的val,完成运算。}
\paragraph{
\hspace{2em}而在stack实现中最复杂的部分只需要两个函数,一个函数用来转化,一个用来计算。其中计算是较为方便实现的,
如果已经实现了转化为后缀表达式,这时只需要按照已经完成的后缀表达式,遇到数字便push stack,遇到
运算符便poptop,因为当初实现后缀表达式时,如果两个运算符连在一起,前者的运算优先级总是高于后者的,
因此在实现时我们需要对加减法,与乘除法的优先级进行定义,遇到左括号时先压入栈,再压入运算符,
而遇到压入的运算符优先级低于或等于时,pop栈顶运算符,若是右括号时则不输出只pop至左括号,中间的运算符均pop且输出
但括号无需输出。最后转化为后缀表达式了之后,将运算数字经过stod转化,依次压入栈内直至遇到第一个运算符,
这是pop栈顶两个元素进行相应运算,并再将运算结果push top。最后直至运算直至没有运算符且只有一个数,
pop并return即可得到答案。
}
\subsection{主要成员函数}
\paragraph{
    \hspace{2em}成员函数:\\
    vector<string> List::transforinto\_postfix();\\
    void judge();\\
    int acquire\_prefrence(char op);
}
\section{测试及运行结果}
\subsection{针对错误输入的测试}
\paragraph{
如下:\\
- 1\$1+5  非法字符\\
- ILLEGAL CHARACTER \$ \\
- 1++2**6 非法运算\\
- ILLEGAL OPERATION \\
- ((2+3+5)*2 括号缺少\\
- ILLEGAL PARENTHESIS MATCHING \\
- \{5+2*(3+4\}+2) 括号匹配问题\\
- ILLEGAL PARENTHESIS MATCHING \\
- 5+2*(3+2*) 子列缺少运算后数字\\
- ILLEGAL OPERATION \\
- 3+5* 缺少运算后数字
- ILLEGAL OPERATION\\
- 5+2*3+(1*\{5+2*(3+4\}+2)+2+1) 子列括号匹配问题\\
- ILLEGAL PARENTHESIS MATCHING \\
- 3+5.2.1(太多点)\\
- ILLEGAL OPERATION\\
- 3e2e3(太多e)\\
- ILLEGAL OPERATION \\
- 2/(2-2)(除0)\\
- A divisor of 0 is ILLEGAL
}
\subsection{针对运算功能的测试}
\paragraph{
如下:\\
- 1+-2.1 (负数运算)\\
- 答案是:-1.1 \textcolor{green}{correct}\\
- -1+2e2 (科学计数法)\\
- 答案是:199 \textcolor{green}{correct}\\
- 123456789*987654321(大数运算)\\
- 答案是:1.21933e+17 \textcolor{green}{correct}\\
- 0.11+2e-2+2E+2(科学计数法再检验及小数)\\
- 答案是:200.13 \textcolor{green}{correct}\\
- 5+2*(3+(1*5+2*(3+4)+2)+2+1)(复杂运算)\\
- 答案是:59 \textcolor{green}{correct}\\
- ((5+3)*2)/(4+(3-1)) {除法且为List难以消去bug类}\\
- 答案是:2.66667 \textcolor{green}{correct}
}
\section{结果及分析}
\paragraph{\hspace{2em}我运行了数次之后,取平均值之后并保留三位小数得到:\newline
}

\par

\begin{tabular}{|c|c|c|} 
    \hline  % 表格上方的横线
     & my heapsort time & std::sort\_heap time \\  % 第一行
    \hline  % 行与行之间的横线
    random sequence & 0.100s & 0.060s \\  % 第二行
    \hline
    ordered sequence & 0.046s & 0.028s \\  % 第三行
    \hline  % 表格底部的横线
    reverse sequence & 0.045s & 0.057s \\
    \hline
    repetitive sequence & 0.082s & 0.056s \\
    \hline
\end{tabular}


\par

\paragraph{\hspace{2em}不难发现,除了reverse sequence,my sortheap都比std效率低
(因为myheapsort是最小堆,最后实现的是逆序,于是我已经将mysortheap结果对调,以保证worst与best
情况对应)我猜测,std存在一些优化策略,比如随机化:
探测若干项是否满足序关系,如果的确如此,那么正如ordered所显示的那样,时间会大大缩短
。相反如果不满足序关系占大多数,std可能会先对此进行一个随机化操作,此时时间复杂度
是$O(1)$不会对整体复杂度产生太大影响。因此std的random,repetitive,reverse的时间
近乎于相等,而我的程序却没有对此的优化,导致不同程序之间运算时间相差过大.除了best情况
剩下的均比std要慢。通过查询资料我了解到,std在运算时可能有并行化向量化减少运算时间,而且
同时优化内存访问模式来提高缓存命中率,这些可能也是std效率比myheapsort快的原因。}

\end{document}
