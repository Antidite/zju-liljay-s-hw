\documentclass[UTF8]{ctexart}
\usepackage{ctex}
\usepackage{geometry, CJKutf8}
\geometry{margin=1.5cm, vmargin={0pt,1cm}}
\setlength{\topmargin}{-1cm}
\setlength{\paperheight}{29.7cm}
\setlength{\textheight}{25.3cm}

% useful packages.
\usepackage{amsfonts}
\usepackage{amsmath}
\usepackage{amssymb}
\usepackage{amsthm}
\usepackage{enumerate}
\usepackage{graphicx}
\usepackage{multicol}
\usepackage{fancyhdr}
\usepackage{layout}
\usepackage{listings}
\usepackage{float, caption}

\lstset{
    basicstyle=\ttfamily, basewidth=0.5em
}

% some common command
\newcommand{\dif}{\mathrm{d}}
\newcommand{\avg}[1]{\left\langle #1 \right\rangle}
\newcommand{\difFrac}[2]{\frac{\dif #1}{\dif #2}}
\newcommand{\pdfFrac}[2]{\frac{\partial #1}{\partial #2}}
\newcommand{\OFL}{\mathrm{OFL}}
\newcommand{\UFL}{\mathrm{UFL}}
\newcommand{\fl}{\mathrm{fl}}
\newcommand{\op}{\odot}
\newcommand{\Eabs}{E_{\mathrm{abs}}}
\newcommand{\Erel}{E_{\mathrm{rel}}}

\begin{document}

\pagestyle{fancy}
\fancyhead{}
\lhead{李伊健, 3230102477}
\chead{数据结构与算法第四次作业}
\rhead{Oct.21th, 2024}

\section{测试程序的设计思路}

我将函数分为三大类
\newline
一.iterator类
\newline 
二.constiterator类
\newline
三.List类
\newline
1.我为了测试List默认构造函数与empty功能,定义空链表none,并测试empty,插入一些值
以
确保空链表可操作。
\newline
2.我为了测试初始化构造函数与右值的pushback与pushfront,我定义并初始化一个链表
l1,并在其中进行pushback与pushfront操作,cout了之后检查其是否正确。
\newline
3.我测试了拷贝构造函数与移动构造函数与赋值构造函数,定义了l1copy与l1move与l2,
并在l1消亡掉后,又验证了他们是深度复制。同时我定义三个临时变量a,b,c来验证了
左值的pushback与pushfront操作。
\newline
4.我通过对l2进行操作
测试了iterator 中begin与end,前后置--与++运算符号以及*运算符,back front函数
size以及== 与!= 函数。
\newline
5.我通过定义lconst并进行操作
测试了constiterator 中begin与end,前后置--与++运算符号以及*运算符,back front函数
size以及== 与!= 函数。
\newline
6.我对l1copy进行了clear操作,并用empty进行验证。
\newline
7.我通过对l2进行操作cout后
验证了popfront与popback操作。
\newline
8.我通过对l2进行操作验证了左右值的insert函数,
以及erase 与erase from to 函数,cout后进行验证。
\newline
9.我借用自己定义的isnull函数判断了两个默认构造函数iterator与constiterator类
默认构造正确。

\section{测试的结果}

测试结果一切正常。
输出结果符合预期。
预期结果为
\newline 1
\newline 0 6 \newline
2 0 1 2 3 4 5 7  \newline
2 0 1 2 3 4 5 7  \newline
2 0 1 2 3 4 5 7 \newline
2 0 1 2 3 4 5 7 \newline
2 0 1 2 3 4 5 7 \newline
2 0 1 2 3 4 5 7 \newline
2 7 7 5 0 0 1 \newline
-1 -2\newline
2 1 -1 0 1 2 3 4 5 -2 3 \newline
11\newline
0 1\newline
a b c d a d\newline
0 1\newline
4\newline
0 \newline
1 -1 0 1 2 3 4 5 -2 \newline
2 6 \newline
1 2 3 4 5 6 7 \newline
1 6 7 6\newline
1 7 7\newline
1 1 \newline

输出结果为:\newline
\newline 1
\newline 0 6 \newline
2 0 1 2 3 4 5 7  \newline
2 0 1 2 3 4 5 7  \newline
2 0 1 2 3 4 5 7 \newline
2 0 1 2 3 4 5 7 \newline
2 0 1 2 3 4 5 7 \newline
2 0 1 2 3 4 5 7 \newline
2 7 7 5 0 0 1 \newline
-1 -2\newline
2 1 -1 0 1 2 3 4 5 -2 3 \newline
11\newline
0 1\newline
a b c d a d\newline
0 1\newline
4\newline
0 \newline
1 -1 0 1 2 3 4 5 -2 \newline
2 6 \newline
1 2 3 4 5 6 7 \newline
1 6 7 6\newline
1 7 7\newline
1 1 \newline

我用 valgrind 进行测试,发现没有发生内存泄露。

\section{(可选)bug报告}

\end{document}
%%% Local Variables: 
%%% mode: latex
%%% TeX-master: t
%%% End: 
