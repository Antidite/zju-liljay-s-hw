\documentclass[UTF8]{ctexart}
\usepackage{ctex}
\usepackage{geometry, CJKutf8}
\geometry{margin=1.5cm, vmargin={0pt,1cm}}
\setlength{\topmargin}{-1cm}
\setlength{\paperheight}{29.7cm}
\setlength{\textheight}{25.3cm}

% useful packages.
\usepackage{amsfonts}
\usepackage{amsmath}
\usepackage{amssymb}
\usepackage{amsthm}
\usepackage{enumerate}
\usepackage{graphicx}
\usepackage{multicol}
\usepackage{fancyhdr}
\usepackage{layout}
\usepackage{listings}
\usepackage{float, caption}

\lstset{
    basicstyle=\ttfamily, basewidth=0.5em
}

% some common command
\newcommand{\dif}{\mathrm{d}}
\newcommand{\avg}[1]{\left\langle #1 \right\rangle}
\newcommand{\difFrac}[2]{\frac{\dif #1}{\dif #2}}
\newcommand{\pdfFrac}[2]{\frac{\partial #1}{\partial #2}}
\newcommand{\OFL}{\mathrm{OFL}}
\newcommand{\UFL}{\mathrm{UFL}}
\newcommand{\fl}{\mathrm{fl}}
\newcommand{\op}{\odot}
\newcommand{\Eabs}{E_{\mathrm{abs}}}
\newcommand{\Erel}{E_{\mathrm{rel}}}

\begin{document}

\pagestyle{fancy}
\fancyhead{}
\lhead{李伊健, 3230102477}
\chead{数据结构与算法第五次作业}
\rhead{November.4th, 2024}

\section{remove功能的实现}
首先我们分为detachMin函数与remove函数两个部分 \newline
\newline
\par
1.通过detachMin函数,我希望将被remove的节点通过指针替换成其右子树的最小节点。
从而通过detachMin函数,我需要剥离该节点,并改变右子树的局部结构,将其剥离后的
结构再连接起来,于是为了保证在不改变引用变量的前提下,改变树的局部结构,减少
引入parent指针的操作,我采用了递归的方式找到右子树的最小节点,并将其赋值给tem
后返回,再通过改变指向其指针为其指向其右节点的指针,成功实现了剥离节点与拼接。\newline

2.回到remove函数,root仅有一个左(右)子节点以及无节点,root为空的情况与示例代码
相同。问题主要在于如何处理两个子节点的情况。首先我先调用detachMin函数,找到右子树
的最小节点,并完善剥离节点处的结构后,用oldNode指向root,后并改变了指向root指针
指向detachMin返回的节点并改变了其左右节点为root的左右节点实现了替换,最终delete
oldNode,完成了对需要remove节点的内存回收。

\section{测试结果及分析}
我创建了bst树,并按照下列顺序插入    bst.insert(10);
bst.insert(5);
bst.insert(15);
bst.insert(3);
bst.insert(7);
bst.insert(12);
bst.insert(18);
bst.insert(1);bst.insert(20)。下面为测试结果:\newline
Initial Tree:\newline
1\newline
3\newline
5\newline
7\newline
10\newline
12\newline
15\newline
18\newline
20\newline
Minimum element: 1\newline
Maximum element: 20\newline
Contains 7? Yes\newline
Contains 20? Yes\newline
Tree after removing 7:(测试删除leaf)\newline
1\newline
3\newline
5\newline
10\newline
12\newline
15\newline
18\newline
20\newline
Tree after removing 15:(测试删除two-children)\newline
1\newline
3\newline
5\newline
10\newline
12\newline
18\newline
20\newline
Tree after removing 10:(测试删除root)\newline
1\newline
3\newline
5\newline
12\newline
18\newline
20\newline
Tree after removing 3:(测试删除one-child)\newline
1\newline
5\newline
12\newline
18\newline
20\newline
Tree after removing 2:(测试删除不存在节点)\newline
1\newline
5\newline
12\newline
18\newline
20\newline
Tree after making empty:\newline
Empty tree\newline
Tree after removing null root:测试删除空树)\newline
Empty tree\newline
Is tree empty? Yes\newline
如同结果和括号注释所示,测试了remove的主要功能以及边界情况,验证成功实现了remove的功能。
\end{document}
