\documentclass[UTF8]{ctexart}
\usepackage{geometry}
\geometry{margin=1.5cm, vmargin={0pt,1cm}}
\setlength{\topmargin}{-1cm}
\setlength{\paperheight}{29.7cm}
\setlength{\textheight}{25.3cm}

% useful packages.
\usepackage{amsfonts}
\usepackage{amsmath}
\usepackage{amssymb}
\usepackage{amsthm}
\usepackage{enumerate}
\usepackage{graphicx}
\usepackage{multicol}
\usepackage{fancyhdr}
\usepackage{layout}
\usepackage{listings}
\usepackage{float, caption}

\lstset{
    basicstyle=\ttfamily, basewidth=0.5em
}

% some common command
\newcommand{\dif}{\mathrm{d}}
\newcommand{\avg}[1]{\left\langle#1 \right\rangle}
\newcommand{\difFrac}[2]{\frac{\dif#1}{\dif#2}}
\newcommand{\pdfFrac}[2]{\frac{\partial#1}{\partial#2}}
\newcommand{\OFL}{\mathrm{OFL}}
\newcommand{\UFL}{\mathrm{UFL}}
\newcommand{\fl}{\mathrm{fl}}
\newcommand{\op}{\odot}
\newcommand{\Eabs}{E_{\mathrm{abs}}}
\newcommand{\Erel}{E_{\mathrm{rel}}}

\begin{document}

\pagestyle{fancy}
\fancyhead{}
\setlength{\headheight}{12.64723pt}
\lhead{李伊健, 3230102477}
\chead{数据结构与算法第六次作业}
\rhead{November.11th, 2024}

\section{remove后保持AVL树功能的实现}
本质上其实就是对于balance的功能实现:\par

\paragraph{
首先我在class BinaryNode 中通过添加height,来进行后续的判断操作。第二步,在构造函数
补充了对height的初始化,同时按照老师上课所讲的旋转balance法,完成了rotateWithLeftChild
和rotateWithRightChild和doubleWithRightChild和doubleWithLeftChild,基本实现了
balance函数的功能,并在insert和remove后加入了balance以保证其最终成为AVL tree.}


\end{document}
